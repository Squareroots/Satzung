\documentclass[a4paper,ngerman]{scrartcl}
\usepackage[utf8]{inputenc}
\usepackage[T1]{fontenc}
\usepackage{babel}
\usepackage{textcomp}
\usepackage{graphicx}
% Section numbering starts with $
\renewcommand*\thesection{\S~\arabic{section}}
% Hyperref allows us to have clickable links in the tables of content (i.e., our todo list)
\usepackage{hyperref}
% Use todo notes
\usepackage{todonotes}
% Allow us to reuse title info later (i.e., reuse the date as \MyDate)
\usepackage{authoraftertitle}

\begin{document}

\title{\includegraphics[width=0.3\textwidth]{sqrts}\\[1cm]Squareroots}
\subtitle{Stealing your flags since 0x7d6}
\date{\todo{Datum bestimmen}XX.XX.2018}
\author{Vereinssatzung}

\maketitle
% Suppress page numbering
\pagenumbering{gobble}
\pagebreak

% --------------------------------------------
% Comment out this block for the final version
% Set tocdepth to 1 so our toc shows
\setcounter{tocdepth}{1}
\listoftodos
\pagebreak
% --------------------------------------------

% Show page numbering and start from 1
% -- this line is superfluous since gobble resets the page number -- \setcounter{page}{1}
\pagenumbering{arabic}

% Show a big "VEREINSSATZUNG"
\textbf{\huge{}Vereinssatzung}{\huge \par}

\section{Name, Sitz, Eintragung, Geschäftsjahr}
\begin{enumerate}
\item Der Verein führt den Namen "`Squareroots"' und soll in das Vereinsregister eingetragen werden. Nach der Eintragung führt er den Zusatz "`e.V."'.
\item Der Verein hat den Sitz in Mannheim.
\item Das Geschäftsjahr ist das Kalenderjahr.
\end{enumerate}

\section{Vereinszweck}
\begin{enumerate}
\item Zweck des Vereins ist Förderung der Bildung und Volksbildung auf dem Gebiet der Informationstechnologien und Informationssicherheit, sowie der Umgang mit diesen.
\item Der Satzungszweck wird insbesondere verwirklicht durch:
\begin{enumerate}
\item Durchführung von öffentlichen, entgeltfreien Veranstaltungen für Computersicherheit, Informationsrecht und kreativen Umgang mit neuen Technologien und deren Anwendungen.
\item Förderung und Unterstützung von Lehre, Forschung, Entwicklung und Aufklärung im Bereich der Informationstechnologien.
\item Förderung der Allgemeinbildung der Bevölkerung im Umgang mit neuen Informationstechnologien und deren Sicherheit.
\item Pflege und Intensivierung des Erfahrungs- und Informationsaustausches zu Themen moderner Informationstechnologien (öffentliche Treffen, Diskussionsforen, Kongresse, Symposien, Tagungen usw.)
% Bemerkung: Von Entropia
\end{enumerate}
\end{enumerate}

\section{Gemeinnützigkeit}
\begin{enumerate}
\item Der Verein verfolgt ausschließlich und unmittelbar gemeinnützige Zwecke im Sinne des Abschnitts "`Steuerbegünstigte Zwecke"' der Abgabenordnung.
\item Der Verein ist selbstlos tätig; er verfolgt nicht in erster Linie eigenwirtschaftliche Zwecke.
% TO Check: Sprittgeld bei Fahrt für Event ist aber ok, oder?
\item Mittel des Vereins dürfen nur für die satzungsmäßigen Zwecke verwendet werden. Die Mitglieder erhalten keine Gewinnanteile und in ihrer Eigenschaft als Mitglieder auch keine sonstigen Zuwendungen aus Mitteln des Vereins.
% Bemerkung: Ein Mitglied kann z.B. nicht als Auslagenerstattung für ein Event 2000€ verlangen
\item Es darf keine Person durch Ausgaben, die dem Zweck des Vereins fremd sind, oder durch unverhältnismäßig hohe Vergütungen begünstigt werden.
\item Alle Inhaber von Vereinsämtern sind unendgeltlich tätig.
\end{enumerate}

\section{Erwerb der Mitgliedschaft}
\begin{enumerate}
\item Die Mitgliedschaft ist möglich als: 
\begin{enumerate}
\item Aktives Mitglied
\item Passives Mitglied
\item Fördermitglied
\item Ehrenmitglied
\end{enumerate}
% Minderjährige Mitglieder sind nur beschränkt geschäftsfähig, daher der Satz mit den Eltern
\item Aktives Mitglied kann jede natürliche Person werden. 
% Bemerkung: Beispiel: Person reist für Jahr in Ausland und möchte in der Zeit nicht Beiträge zahlen. Bedeutet für uns, dass wir passive Mitglieder für MVs auch nicht einladen müssen. Nur natürliche Personen erlaubt, da Firmen und Co ja Fördermitglieder sein würden
\item Ein passives Mitglied ist eine natürliche Person mit ruhender Mitgliedschaft. Dieses Mitglied hat keinerlei Anspruch auf Vereinsleistungen. Zudem besteht kein Stimmrecht bei Mitgliederversammlungen.
\item Fördermitglied kann jede juristische oder natürliche Person sein. Mit der Fördermitgliedschaft ist kein Stimmrecht verbunden.
% Bemerkung: Idee (siehe auch Wikipedia): Wenn Leute wie Matthias geehrt werden sollen, die das Team gegründet haben. Ist einfach nur Symbolische Ehrung ohne Rechte und Pflichten
\item 
Personen, die sich um den Verein besonders verdient gemacht haben, können auf Vorschlag des Vorstandes durch Beschluss der Mitgliederversammlung zu Ehrenmitgliedern ernannt werden.
Ehrenmitglied kann jede natürliche Person sein. Mit der Ehrenmitgliedschaft ist kein Stimmrecht verbunden.
\item Über den Antrag auf Aufnahme in den Verein entscheidet der Vorstand. Eine Ablehnung des Aufnahmeantrages ist ohne Angabe von Gründen möglich. Gegen die Entscheidung kann die Mitgliederversammlung angerufen werden.
\item Die Mitgliedschaft beginnt nach positivem Aufnahmebescheid mit dem Eingang des Aufnahmebeitrages und des ersten Mitgliedsbeitrages. Es ist ein schriftlicher Aufnahmeantrag an den Verein zu richten.
\item Bei Minderjährigen ist die Zustimmung des gesetzlichen Vertreters erforderlich.
\end{enumerate}

\section{Beendigung der Mitgliedschaft}
\begin{enumerate}
\item Die Mitgliedschaft endet durch Austritt, Ausschluss, Tod der natürlichen Person bzw. Auflösung der juristischen Person.
\item Der Austritt eines Mitgliedes ist mit einer Kündigungsfrist von 2 Monaten zum Quartalsende möglich. Er erfolgt durch schriftliche Erklärung gegenüber dem Vorstand.
\item Aus dem Verein kann durch den Vorstand mit einfacher Mehrheit, mit sofortiger Wirkung ausgeschlossen werden: 
\begin{enumerate}
\item wer gegen die Ziele und Interessen des Vereins grob fahrlässig oder vorsätzlich verstoßen hat. 
\item wer trotz Mahnung mit dem Beitrag für drei Monate im Rückstand bleibt.
\item wer der Satzung des Vereins zuwiderhandelt.
\item wer sich grob unsozial verhält oder das Ansehen des Vereins nachhaltig schädigt.
\item wenn ein sonstiger wichtiger Grund vorliegt.
\end{enumerate}
% Bemerkung: Das hier gilt vor dem Beschluss und soll dem Mitglied Möglichkeit geben, um den Vorfall zu erklären und somit ggf. den Vorstand umzustimmen
\item Dem Mitglied muss vor der Beschlussfassung Gelegenheit zur Rechtfertigung bzw. Stellungnahme gegeben werden. Hierzu ist das Mitglied unter Einhaltung einer Mindestfrist von zehn Tagen schriftlich aufzufordern.
% Bemerkung: Wenn der Vorstand trotzdem den Ausschluss für notwendig hält, kann das Mitglied Wiederspruch einlegen, um die MV entscheiden zu lassen. Ansonsten fliegt es direkt raus
\item Gegen den Ausschließungsbeschluss kann innerhalb einer Frist von zwei Wochen nach Mitteilung des Ausschlusses schriftlich Widerspruch beim Vorstand eingelegt werden, über den die nächste ordentliche oder außerordentliche Mitgliederversammlung mit absoluter Mehrheit entscheidet. Bis zur Entscheidung der Mitgliederversammlung ist die Mitgliedschaft des betreffenden Mitgliedes nach §4 Absatz 3 eine passive Mitgliedschaft.
%TODO: Querverweis auf passive Mitgliedschaft
\item Bei Beendigung der Mitgliedschaft, gleich aus welchem Grund, erlöschen alle Ansprüche aus dem Mitgliedschaftsverhältnis. Noch ausstehende Verpflichtungen aus dem Mitgliedschaftsverhältnis, insbesondere ausstehende Beitragspflichten, bleiben hiervon unberührt. Vereinseigene Gegenstände sind dem Verein herauszugeben. 
% Kommentar: GGf. folgenden Satz (nach CHeck der Beitragsordnung) streichen
Dem austretenden Mitglied steht kein Anspruch auf Rückzahlung überzahlter Beiträge zu.
\end{enumerate}

\section{Beiträge}
\begin{enumerate}
% Bemerkung: Umlagen => z.B. wenn wir uns was teueres kaufen wollen, was unser aktuelles Budget übersteigt und die Kosten daher auf alle Mitglieder verteilen.
% Gebühren für besondere Leistungen => z.B. Hotel/Bahnbuchungen für einige Mitglieder (die über die Gebühren letztendlich selbst bezahlen müssen), um Gruppenrabatte nutzen zu können
% TODO: Umlagengeschichte und die Frage nach Bezahlung im Voraus muss in die Beitragsordnung
\item Der Verein erhebt einen Aufnahmebeitrag und einen regelmäßigen Mitgliedsbeitrag. Es können zusätzlich Umlagen, Gebühren für besondere Leistungen des Vereins erhoben werden.
\item Für die Regelung der Beiträge, Umlagen und Gebühren beschließt die Mitgliederversammlung eine Beitragsordnung.
% In Beitragsordnung: Das Mitglied ist verpflichtet, dem Verein Änderungen der Bankverbindung, der Anschrift sowie der Mailadresse mitzuteilen.
\end{enumerate}

\section{Pflichten}
\begin{enumerate}
\item Jedes Mitglied ist verpflichtet, sich nach der Satzung und den weiteren Ordnungen des Vereins zu verhalten. Alle Mitglieder sind zu gegenseitiger Rücksichtnahme verpflichtet.
\end{enumerate}

\section{Organe des Vereins}
\begin{enumerate}
\item Organe des Vereins sind
\begin{enumerate}
\item der Vorstand
\item die Mitgliederversammlung
\end{enumerate}
\end{enumerate}

% TODO: Innerhalb des Vorstands einigen
\section{Der Vorstand}
\begin{enumerate}
\item Der Vorstand besteht aus:
\begin{enumerate}
\item einem Vorsitzenden
\item einem Stellvertreter
\item einem Kassenwart
\end{enumerate}
% TODO: Stean fragt beim C3MA nach, was es mit "allein vertretungsberechtigt" auf sich hat (war das ein Requirement für Banking?). Sh4dow fragt bei Bank nach, ob das ein Requirement ist
\item Er vertritt den Verein gerichtlich und außergerichtlich. Jedes Vorstandsmitglied ist allein vertretungsberechtigt.
\item Der Vorstand wird von der Mitgliederversammlung für die Dauer von zwei Jahren gewählt. Die Wiederwahl der Vorstandsmitglieder ist möglich.
\item Der Vorsitzende wird von der Mitgliederversammlung in einem besonderen Wahlgang bestimmt. Die jeweils amtierenden Vorstandsmitglieder bleiben nach Ablauf ihrer Amtszeit im Amt, bis Nachfolger gewählt sind.
\item Dem Vorstand obliegt die Führung der laufenden Geschäfte des Vereins. Er hat insbesondere folgende Aufgaben:
\begin{enumerate}
\item Der Vorstand kann für die Geschäfte der laufenden Verwaltung einen Geschäftsführer bestellen. Dieser ist berechtigt, an den Sitzungen des Vorstandes mit beratender Stimme teilzunehmen.
\item Der Vorstand fasst seine Beschlüsse mit einfacher Mehrheit.
\end{enumerate}
\item Beschlüsse des Vorstands können bei Eilbedürftigkeit auch schriftlich oder fernmündlich gefasst werden, wenn alle Vorstandsmitglieder ihre Zustimmung zu diesem Verfahren schriftlich oder fernmündlich erklären. Schriftlich oder fernmündlich gefasste Vorstandsbeschlüsse sind schriftlich niederzulegen und von allen Vorstandsmitgliedern zu unterzeichnen.
\item Ein Vorstandsmitglied ist nicht berechtigt ein anderes Vorstandsmitglied in Vorstandssitzungen zu vertreten.
\end{enumerate}

\section{Zuständigkeiten des Vorstands}
\begin{enumerate}
\item Der Vorstand führt die Geschäfte des Vereins und fasst die erforderlichen Beschlüsse.
\item Er ist zu rechtsgeschäftlichen Verpflichtungen zu Lasten des Vereins bis zu einer Höhe von EUR 313,37 ermächtigt. Diese Bestimmung betrifft das Innenverhältnis. Verpflichtungen, welche die Höhe von EUR 313,37 übersteigen, müssen vom Vorstand einstimmig genehmigt und beschlossen werden.
\item In dringenden, keinen Aufschub duldenden Dingen kann der Vorstand mit der Zustimmung aller Vorstandsmitglieder über diese Befugnisse hinaus handeln. Diese Bestimmung betrifft das Innenverhältnis. Er ist verpflichtet die Mitglieder hierüber unverzüglich zu informieren.
\item Der Vorstand gibt sich eine Geschäftsordnung. Diese ist den Mitgliedern innerhalb einer Woche schriftlich oder per E-Mail zur Verfügung zu stellen.
\end{enumerate}

\section{Mitgliederversammlung}
\begin{enumerate}
\item Die Mitgliederversammlung ist einmal jährlich einzuberufen.
\item Eine außerordentliche Mitgliederversammlung ist einzuberufen, wenn es das Vereinsinteresse erfordert oder wenn die Einberufung von mindestens 20\% der Vereinsmitglieder schriftlich und unter Angabe des Zweckes und der Gründe verlangt wird.
\item Die Einberufung der Mitgliederversammlung erfolgt schriftlich oder in elektronischer Form durch den Vorstand unter Wahrung einer Einladungsfrist von mindestens zwei Wochen bei gleichzeitiger Bekanntgabe der Tagesordnung. Die Frist beginnt mit dem auf die Absendung des Einladungsschreibens folgenden Tag. Das Einladungsschreiben gilt dem Mitglied als zugegangen, wenn es an die letzte vom Mitglied des Vereins schriftlich bekannt gegebene Adresse gerichtet ist.
\item Jedes stimmberechtigte Vereinsmitglied hat das gleiche Stimmgewicht.
\item Stimmberechtigt ist jedes anwesende, aktive Mitglied.
\item Über die Beschlüsse der Mitgliederversammlung ist Protokoll zu führen, das vom Versammlungsleiter zu unterzeichnen ist.
\item Die Mitgliederversammlung als das oberste beschlussfassende Vereinsorgan ist grundsätzlich für alle Aufgaben zuständig, sofern bestimmte Aufgaben gemäß dieser Satzung nicht einem anderen Vereinsorgan übertragen wurden. Ihr sind insbesondere die Jahresrechnung und der Jahresbericht zur Beschlussfassung über die Genehmigung und die Entlastung des Vorstandes schriftlich vorzulegen. Die Mitgliederversammlung entscheidet z. B. auch über
\begin{enumerate}
\item Gebührenbefreiungen,
\item Aufgaben des Vereins,
\item Aufnahme von Darlehen ab EUR 313,37
\item Genehmigung aller Geschäftsordnungen für den Vereinsbereich,
\item Satzungsänderungen,
\item Auflösung des Vereins.
\end{enumerate}
Die Mitgliederversammlung ist nicht berechtigt über 
\begin{enumerate}
\item die Abschaffung der Beitragsordnung,
\item \todo{Rechte der Mitgliederversammlung}\dots{} zu entscheiden.
\end{enumerate}
\item Eine Vertretung eines abwesenden Mitglieds durch ein anderes ist möglich, wenn die Vertretungsbefugnis in Textform nachgewiesen wird. Jedes anwesende Mitglied kann höchstens zwei abwesende Mitglieder vertreten.
\item Jede satzungsmäßig einberufene Mitgliederversammlung wird als beschlussfähig anerkannt ohne Rücksicht auf die Zahl der erschienenen Vereinsmitglieder.
\item Die Mitgliederversammlung fasst ihre Beschlüsse mit einfacher Mehrheit. Bei Stimmengleichheit gilt ein Antrag als abgelehnt.
\end{enumerate}

\section{Beurkundung von Beschlüssen}
\begin{enumerate}
\item Die in Vorstandssitzungen und in Mitgliederversammlungen erfassten Beschlüsse sind schriftlich niederzulegen und vom Vorstand zu unterzeichnen.
\end{enumerate}

\section{Satzungsänderung}
\begin{enumerate}
\item Für Satzungsänderungen ist eine 3/4-Mehrheit der anwesenden, stimmberechtigten Vereinsmitglieder erforderlich. Stimmenthaltungen bleiben außer Betracht.
 Für Änderungen des Satzungszwecks ist eine Mehrheit von \todo{Definiere benötigte Mehrheit für Änderung von Satzungszweck}\dots{} der erschienenen Vereinsmitglieder erforderlich. Über Satzungsänderungen kann in der Mitgliederversammlung nur abgestimmt werden, wenn auf diesen Tagesordnungspunkt bereits in der Einladung zur Mitgliederversammlung hingewiesen wurde und der Einladung sowohl der bisherige als auch der vorgesehene neue Satzungstext beigefügt worden waren.
\item Satzungsänderungen, die von Aufsichts-, Gerichts- oder Finanzbehörden aus formalen Gründen verlangt werden, kann der Vorstand von sich aus vornehmen. Diese Satzungsänderungen müssen allen Vereinsmitgliedern zeitnah schriftlich mitgeteilt werden.
\item Zusätzliche berechtigte Stimmen können durch die Übertragung der Stimme eines ordentlichen Mitglieds mittels einer Vollmacht auf ein teilnehmendes ordentliches Mitglied erfolgen. Ein teilnehmendes ordentliches Mitglied kann maximal zwei weitere Stimmen zusätzlich zur eigenen auf sich vereinen.
\end{enumerate}

\section{Vereinsordnungen}
\begin{enumerate}
\item Soweit die Satzung nicht etwas Abweichendes regelt ist der geschäftsführende Vorstand ermächtigt durch Beschluss nachfolgende Ordnungen zu erlassen:
\begin{enumerate}
\item Beitragsordnung
\item Finanzordnung
\item Geschäftsordnung für den geschäftsführenden Vorstand und den Gesamtvorstand.
\end{enumerate}
\item Die Finanzordnung und die Geschäftsordnung sind nicht Bestandteil der Satzung.
\end{enumerate}

\section{Haftung des Vereins}
\begin{enumerate}
\item Ehrenamtlich Tätige und Organ- oder Amtsträger, deren Vergütung 720,00
€ im Jahr nicht übersteigt, haften für Schäden gegenüber den Mitgliedern
und gegenüber dem Verein, die sie in Erfüllung ihrer ehrenamtlichen
Tätigkeit verursachen, nur für Vorsatz und grobe Fahrlässigkeit.
\item Der Verein haftet gegenüber den Mitgliedern im Innenverhältnis nicht für fahrlässig verursachte Schäden, die Mitglieder bei der Benutzung von Anlagen oder Einrichtungen des Vereins oder bei Vereinsveranstaltungen erleiden, soweit solche Schäden nicht durch Versicherungen des Vereins abgedeckt sind.
\end{enumerate}

\section{Datenschutz im Verein}
\begin{enumerate}
\item Zur Erfüllung der Zwecke und Aufgaben des Vereins werden unter Beachtung der gesetzlichen Vorgaben des Bundesdatenschutzgesetzes (BDSG) personenbezogene Daten über persönliche und sachliche Verhältnisse der Mitglieder im Verein genutzt, gespeichert, übermittelt und verändert.
\item Jedes Vereinsmitglied hat das Recht auf:
\begin{enumerate}
\item Auskunft über die zu seiner Person gespeicherten Daten;
\item Berichtigung über die zu seiner Person gespeicherten Daten, wenn sie unrichtig sind;
\item Sperrung der zu seiner Person gespeicherten Daten, wenn sich bei behaupteten Fehlern weder deren Richtigkeit noch deren Unrichtigkeit feststellen lässt;
\item Löschung der zu seiner Person gespeicherten Daten, wenn die Speicherung unzulässig war.
\end{enumerate}
\item Den Organen des Vereins, allen Mitarbeitern oder sonst für den Verein Tätigen ist es untersagt, personenbezogene Daten unbefugt zu anderen als dem jeweiligen zur Aufgabenerfüllung gehörenden Zweck zu verarbeiten, bekannt zu geben, Dritten zugänglich zu machen oder sonst zu nutzen. Diese Pflicht besteht auch über das Ausscheiden der oben genannten Personen aus dem Verein hinaus.
\end{enumerate}

\section{Auflösung des Vereins und Vermögensbindung}
\begin{enumerate}
\item Die Auflösung des Vereins kann nur in einer zu diesem Zweck einberufenen Mitgliederversammlung beschlossen werden. Zur Auflösung des Vereins ist eine Mehrheit von zwei Drittel/drei Viertel/vier Fünftel der abgegebenen gültigen Stimmen erforderlich.
\item Sofern die Mitgliederversammlung nicht anderes beschließt, sind im Falle der Auflösung die Vorstandsmitglieder als die Liquidatoren des Vereins bestellt.
\item Bei Auflösung des Vereins oder bei Wegfall steuerbegünstigter Zwecke fällt das Vermögen des Vereins:
\begin{enumerate}
\item an den/die/das \todo{Hier fehlt auch etwas}\dots{} (Bezeichnung einer juristischen Person des öffentlichen Rechts oder einer anderen steuerbegünstigten Körperschaft), der/die/das es unmittelbar und ausschließlich für gemeinnützige, mildtätige oder kirchliche Zwecke zu verwenden hat, oder
\item an eine juristische Person des öffentlichen Rechts oder eine andere steuerbegünstigte Körperschaft zwecks Verwendung für \todo{Definiere gemeinnützigen Zweck}\dots{}
(Angabe eines bestimmten gemeinnützigen Zwecks)
\item in Teilen von 25 \% an den Vorstand, 25 \% an die noch aktiven Mitglieder und 50 \% an die Universität Mannheim. Reine Ehrenmitglieder und Fördermitglieder werden in dieser Aufteilung nicht als aktive Mitglieder betrachtet und sind somit von der Verteilung ausgeschlossen.
\end{enumerate}
\item Die Auflösung wird dem Vereinsregister durch den Vorstand mitgeteilt.
\end{enumerate}

\section{Gültigkeit dieser Satzung}
\begin{enumerate}
\item Diese Satzung wurde durch die Mitgliederversammlung am \MyDate beschlossen
\item Diese Satzung tritt mit Eintragung in das Vereinsregister in Kraft.
\item Alle bisherigen Satzungen treten zu diesem Zeitpunkt damit außer Kraft.
\item Wird ein Paragraph der Satzung für ungültig erklärt, bleibt die Gültigkeit der anderen Satzungsparagraphen davon unberührt.
\end{enumerate}

\section{Sonstiges}
\begin{enumerate}
\item Jedes Mitglied muss smu, Sh4dow und Stean vor jedem Treffen eine Mate ausgeben
\end{enumerate}
\end{document}
