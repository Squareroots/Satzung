\documentclass[a4paper,ngerman]{scrartcl}
\usepackage[utf8]{inputenc}
\usepackage[T1]{fontenc}
\usepackage{babel}
\usepackage{textcomp}
\usepackage{graphicx}
\usepackage{nameref}
\usepackage{smartref}
\addtoreflist{section}
% Section numbering starts with $
\renewcommand*\thesection{\S~\arabic{section}}
% Hyperref allows us to have clickable links in the tables of content (i.e., our todo list)
\usepackage{hyperref}
% Use todo notes
\usepackage{todonotes}
% Allow us to reuse title info later (i.e., reuse the date as \MyDate)
\usepackage{authoraftertitle}

\newcommand{\customref}[1]{\sectionref{#1}, Absatz \ref{#1}}

\begin{document}

\title{\includegraphics[width=0.3\textwidth]{sqrts}\\[1cm]Squareroots}
\subtitle{Stealing your flags since 0x7d6}
\date{14.11.2019}
\author{Vereinssatzung}

\maketitle
% Suppress page numbering
\pagenumbering{gobble}
\pagebreak

% --------------------------------------------

% Show page numbering and start from 1
% -- this line is superfluous since gobble resets the page number -- \setcounter{page}{1}
\pagenumbering{arabic}

% Show a big "VEREINSSATZUNG"
\textbf{\huge{}Vereinssatzung}{\huge \par}

\section{Name, Sitz, Eintragung, Geschäftsjahr}
\begin{enumerate}
\item Der Verein führt den Namen "`Squareroots"' und soll in das Vereinsregister eingetragen werden.
\item Der Verein hat den Sitz in Mannheim.
\item Das Geschäftsjahr ist das Kalenderjahr.
\end{enumerate}

\section{Vereinszweck}
\begin{enumerate}
\item Zweck des Vereins ist Förderung der Volksbildung auf dem Gebiet der Informationstechnologien und Informationssicherheit.
\item Der Satzungszweck wird insbesondere verwirklicht durch:
\begin{enumerate}
\item Durchführung von öffentlichen, entgeltfreien Veranstaltungen für Computersicherheit, Informationsrecht und kreativen Umgang mit neuen Technologien und deren Anwendungen.
\item Förderung und Unterstützung von Lehre, Forschung, Entwicklung und Aufklärung im Bereich der Informationstechnologien.
\item Förderung der Allgemeinbildung der Bevölkerung im Umgang mit neuen Informationstechnologien und deren Sicherheit.
\item Pflege und Intensivierung des Erfahrungs- und Informationsaustausches zu Themen moderner Informationstechnologien (öffentliche Treffen, Diskussionsforen, Kongresse, Symposien, Tagungen usw.).
% Bemerkung: Von Entropia
\end{enumerate}
\end{enumerate}

\section{Gemeinnützigkeit}
\begin{enumerate}
\item Der Verein verfolgt ausschließlich und unmittelbar gemeinnützige Zwecke im Sinne des Abschnitts "`Steuerbegünstigte Zwecke"' der Abgabenordnung.
\item Der Verein ist selbstlos tätig; er verfolgt nicht in erster Linie eigenwirtschaftliche Zwecke.
% TO Check: Sprittgeld bei Fahrt für Event ist aber ok, oder?
\item Mittel des Vereins dürfen nur für die satzungsmäßigen Zwecke verwendet werden. Die Mitglieder erhalten keine Gewinnanteile und in ihrer Eigenschaft als Mitglieder auch keine sonstigen Zuwendungen aus Mitteln des Vereins.
% Bemerkung: Ein Mitglied kann z.B. nicht als Auslagenerstattung für ein Event 2000€ verlangen
\item Es darf keine Person durch Ausgaben, die dem Zweck des Vereins fremd sind, oder durch unverhältnismäßig hohe Vergütungen oder Aufwandsentschädigung begünstigt werden.
\item Alle Inhaber von Vereinsämtern sind unentgeltlich tätig.
\end{enumerate}

\section{Erwerb der Mitgliedschaft}
\begin{enumerate}
\item Es ist ein schriftlicher Aufnahmeantrag an den Verein zu richten.
\item Die Mitgliedschaft ist möglich als:
\begin{enumerate}
\item Aktives Mitglied
\item Passives Mitglied
\item Fördermitglied
\item Ehrenmitglied
\end{enumerate}
% Minderjährige Mitglieder sind nur beschränkt geschäftsfähig, daher der Satz mit den Eltern
\item Aktives Mitglied kann jede natürliche Person werden.
% Bemerkung: Beispiel: Person reist für Jahr in Ausland und möchte in der Zeit nicht Beiträge zahlen. Bedeutet für uns, dass wir passive Mitglieder für MVs auch nicht einladen müssen. Nur natürliche Personen erlaubt, da Firmen und Co ja Fördermitglieder sein würden
\item Ein passives Mitglied ist eine natürliche Person mit ruhender Mitgliedschaft. Dieses Mitglied hat keinerlei Anspruch auf Vereinsleistungen. Zudem besteht kein Stimmrecht bei Mitgliederversammlungen.\label{passive Mitgliedschaft}
\item Fördermitglied kann jede juristische oder natürliche Person sein. Mit der Fördermitgliedschaft ist kein Stimmrecht verbunden.
% Bemerkung: Idee (siehe auch Wikipedia): Wenn Leute wie Matthias geehrt werden sollen, die das Team gegründet haben. Ist einfach nur Symbolische Ehrung ohne Rechte und Pflichten
\item
Personen, die sich um den Verein besonders verdient gemacht haben, können auf Vorschlag des Vorstandes durch Beschluss der Mitgliederversammlung zu Ehrenmitgliedern ernannt werden.
Ehrenmitglied kann jede natürliche Person sein. Mit der Ehrenmitgliedschaft ist kein Stimmrecht verbunden.
\item Über den Antrag auf Aufnahme in den Verein entscheidet der Vorstand. Eine Ablehnung des Aufnahmeantrages ist ohne Angabe von Gründen möglich. Gegen die Entscheidung kann die Mitgliederversammlung angerufen werden.
\item Die Mitgliedschaft beginnt nach positivem Aufnahmebescheid mit dem Eingang des Aufnahmebeitrages und des ersten Mitgliedsbeitrages.
% Schriftform = Auf Papier!
\item Bei Minderjährigen ist die Zustimmung des gesetzlichen Vertreters erforderlich.
\end{enumerate}

\section{Beendigung der Mitgliedschaft}
\begin{enumerate}
\item Die Mitgliedschaft endet durch Austritt, Ausschluss, Tod der natürlichen Person bzw. Auflösung der juristischen Person.
\item Der Austritt eines Mitgliedes ist mit einer Kündigungsfrist von 2 Monaten zum Quartalsende möglich. Er erfolgt durch schriftliche Erklärung gegenüber dem Vorstand.
% auch hier gilt: schriftlich = Papierform!
\item Aus dem Verein kann durch den Vorstand mit einfacher Mehrheit, mit sofortiger Wirkung ausgeschlossen werden:
\begin{enumerate}
\item wer gegen die Ziele und Interessen des Vereins grob fahrlässig oder vorsätzlich verstoßen hat.
\item wer trotz Mahnung mit dem Beitrag für drei Monate im Rückstand bleibt.
\item wer der Satzung des Vereins zuwiderhandelt.
\item wer sich grob unsozial verhält oder das Ansehen des Vereins nachhaltig schädigt.
\item wenn ein sonstiger wichtiger Grund vorliegt.
\end{enumerate}
% Bemerkung: Das hier gilt vor dem Beschluss und soll dem Mitglied Möglichkeit geben, um den Vorfall zu erklären und somit ggf. den Vorstand umzustimmen
\item Dem Mitglied muss vor der Beschlussfassung Gelegenheit zur Rechtfertigung bzw. Stellungnahme gegeben werden. Hierzu ist das Mitglied unter Einhaltung einer Mindestfrist von zehn Tagen schriftlich aufzufordern.
% Bemerkung: Wenn der Vorstand trotzdem den Ausschluss für notwendig hält, kann das Mitglied Widerspruch einlegen, um die MV entscheiden zu lassen. Ansonsten fliegt es direkt raus
% schriftlich = Auf Papier mit Einschreiben
\item Gegen den Ausschließungsbeschluss kann innerhalb einer Frist von zwei Wochen nach Mitteilung des Ausschlusses schriftlich Widerspruch beim Vorstand eingelegt werden, über den die nächste ordentliche oder außerordentliche Mitgliederversammlung mit absoluter Mehrheit entscheidet. Bis zur Entscheidung der Mitgliederversammlung ist die Mitgliedschaft des betreffenden Mitgliedes nach \customref{passive Mitgliedschaft} eine passive Mitgliedschaft.
\item Bei Beendigung der Mitgliedschaft, gleich aus welchem Grund, erlöschen alle Ansprüche aus dem Mitgliedschaftsverhältnis. Noch ausstehende Verpflichtungen aus dem Mitgliedschaftsverhältnis, insbesondere ausstehende Beitragspflichten, bleiben hiervon unberührt. Vereinseigene Gegenstände sind dem Verein herauszugeben.
% Kommentar: GGf. folgenden Satz (nach Check der Beitragsordnung) streichen
Dem austretenden Mitglied steht kein Anspruch auf Rückzahlung überzahlter Beiträge zu.
\end{enumerate}

\section{Beiträge}
\begin{enumerate}
% Bemerkung: Umlagen => z.B. wenn wir uns was teueres kaufen wollen, was unser aktuelles Budget übersteigt und die Kosten daher auf alle Mitglieder verteilen.
% Gebühren für besondere Leistungen => z.B. Hotel/Bahnbuchungen für einige Mitglieder (die über die Gebühren letztendlich selbst bezahlen müssen), um Gruppenrabatte nutzen zu können
\item Der Verein erhebt einen Aufnahmebeitrag, der in der Beitragsordnung definiert ist. Es können zusätzlich Umlagen, Gebühren oder Aufwandentschädigungen für besondere Leistungen des Vereins erhoben werden.
\item Für die Regelung der Beiträge, Umlagen und Gebühren beschließt die Mitgliederversammlung eine Beitragsordnung. \label{Beschluss einer Beitragsordnung}
% In Beitragsordnung: Das Mitglied ist verpflichtet, dem Verein Änderungen der Bankverbindung, der Anschrift sowie der Mailadresse mitzuteilen.
\end{enumerate}

\section{Pflichten}
\begin{enumerate}
\item Jedes Mitglied ist verpflichtet, sich nach der Satzung und den weiteren Ordnungen des Vereins zu verhalten. Alle Mitglieder sind zu gegenseitiger Rücksichtnahme verpflichtet.
\end{enumerate}

\section{Organe des Vereins}
\begin{enumerate}
\item Organe des Vereins sind
\begin{enumerate}
\item der Vorstand
\item die Mitgliederversammlung
\end{enumerate}
\end{enumerate}

\section{Der Vorstand}
\begin{enumerate}
\item Der Vorstand besteht aus:
\begin{enumerate}
\item einem Vorsitzenden
\item einem Stellvertreter
\item einem Kassenwart
\end{enumerate}
\item Jedes Vorstandsmitglied ist allein vertretungsberechtigt. Im Innenverhältnis ist gegenüber Dritten abzugebenden Rechtshandlung ein Vorstandsbeschluss zu erlassen.
% Sh4dow hat bei Bank nachgefragt: "Jedes Vorstandsmitglied ist allein vertretungsberechtigt" -> Dieser Satz spielt vor allem bezgl. Online-Banking eine Rolle -> sonst müssen für jede Überweisung stets 2 Vorstandsmitglieder zustimmen bzw. ggf. sogar vor Ort sein
% Der letzte Satz beschränkt den Handlungsspielraum des vorherigen Satzes wiederum, so dass potentielle "Alleingänge" eingeschränkt werden.
% Durch das BGB §26 ist außerdem abgedeckt: "Er vertritt den Verein gerichtlich und außergerichtlich"
\item Der Vorstand wird von der Mitgliederversammlung für die Dauer von zwei Jahren gewählt. Die Wiederwahl der Vorstandsmitglieder ist möglich.
\item Die Mitglieder des Vorstands sind grundsätzlich ehrenamtlich tätig.
\item Die jeweils amtierenden Vorstandsmitglieder bleiben nach Ablauf ihrer Amtszeit im Amt, bis Nachfolger gewählt sind.
\item Vorstandsmitglied kann nur werden, wer mindestens ein Jahr Vereinsmitglied ist.
% Haben wir uns überlegt, weil ein Vorstandsmitglied mit den Gegebenheiten und der Mentalität im Verein vertraut sein sollte
\item Mit der Beendigung der Vereinsmitgliedschaft endet auch die Mitgliedschaft im Vereinsvorstand. Scheidet ein Mitglied des Vorstands vorzeitig aus, so ist binnen eines
Monats eine außerordentliche Mitgliederversammlung einzuberufen und eine Wahl für die Besetzung des vakanten Vorstandspostens durchzuführen.
\item Ein Vorstandsmitglied ist nicht berechtigt ein anderes Vorstandsmitglied in Vorstandssitzungen zu vertreten.
\end{enumerate}

\section{Zuständigkeiten des Vorstands}
\begin{enumerate}
\item Der Vorstand führt die Geschäfte des Vereins und fasst die erforderlichen Beschlüsse. Er hat insbesondere folgende Aufgaben:
\begin{enumerate}
\item Der Vorstand kann für die Geschäfte der laufenden Verwaltung einen ehrenamtlichen Geschäftsführer bestellen. Dieser ist berechtigt, an den Sitzungen des Vorstandes mit beratender Stimme teilzunehmen.
\item Der Vorstand fasst seine Beschlüsse mit einfacher Mehrheit.
\end{enumerate}
% Haben wir jetzt nicht aufgenommen, da auch in Satzung von C3MA und RZL nicht drinnen, dafür aber in anderen Mustern:
% * die Führung der Bücher sowie die Erstellung des Jahresabschlusses
% * die Ausführung von Beschlüssen der Mitgliederversammlung,
% * die Verwaltung des Vereinsvermögens und die Anfertigung des Jahresberichts
% * die Aufnahme neuer Mitglieder
\item Er ist zu rechtsgeschäftlichen Verpflichtungen zu Lasten des Vereins bis zu einer Höhe von EUR 313,37 ermächtigt. Verpflichtungen, welche die Höhe von EUR 313,37 übersteigen, müssen vom Vorstand einstimmig genehmigt und beschlossen werden.
% Andererseits muss sich der Vorstand natürlich spätestens bei der MV für seine Ausgaben rechtfertigen, wobei die Mitglieder den Vorstand entlasten oder nicht können.
\item In dringenden, keinen Aufschub duldenden Dingen kann der Vorstand mit der Zustimmung aller Vorstandsmitglieder über diese Befugnisse hinaus handeln. Diese Bestimmung betrifft das Innenverhältnis. Er ist verpflichtet die Mitglieder hierüber unverzüglich in Textform zu informieren.
% in Textform => auch elektronisch wie E-Mail ist erlaubt. Es muss allerdings dauerhaft sein, womit z.B. Slack wegfällt. Genauere Rechtsform:
% # Die Textform und E-Mails
% Anders ist das bei der – deutlich weniger bekannten – Textform. Für diese Form genügt gem. § 126b BGB auch eine andere dauerhafte Wiedergabe in Schriftzeichen. Es muss auch keine eigenhändige Unterschrift vorliegen; es genügt, wenn der Name genannt und irgendwie deutlich wird, wo die Erklärung endet. Diese Anforderungen lassen sich durch eine E-Mail erfüllen. Die Textform wird jedoch im Gesetz nur an wenigen Stellen für ausreichend gehalten. Immerhin die Information des Kunden bei Verbraucherverträgen oder der Wiederruf des Verbrauchers bei einem Internetgeschäft können in Textform – also per E-Mail – erfolgen.
\item Der Vorstand kann sich eine Geschäftsordnung geben, in der u.a. die Aufgabenbereiche der einzelnen Vorstandsmitglieder festgelegt werden.
\item Die Beschlüsse des Vorstands sind zu protokollieren
\end{enumerate}

\section{Mitgliederversammlung}
\begin{enumerate}
\item Die Mitgliederversammlung ist einmal jährlich einzuberufen.
\item Eine außerordentliche Mitgliederversammlung ist einzuberufen, wenn es das Vereinsinteresse erfordert oder wenn die Einberufung von mindestens 20\% der Vereinsmitglieder in Textform und unter Angabe des Zweckes und der Gründe verlangt wird.
% auch hier gilt: Textform = E-Mail oder andere dauerhafte Wiedergabe in Schriftzeichen
% Wir (der Vorstand) versuchen es hier den potentiellen 20% einfach zu ermöglichen....
% Vorstand kann auch in den 20% der anfragenden Mitglieder enthalten sein
\item Die Einberufung der Mitgliederversammlung erfolgt in Textform durch den Vorstand unter Wahrung einer Einladungsfrist von mindestens zwei Wochen bei gleichzeitiger Bekanntgabe der Tagesordnung. Die Frist beginnt mit dem auf die Absendung des Einladungsschreibens folgenden Tag. Das Einladungsschreiben gilt dem Mitglied auch als zugegangen, wenn es an die letzten vom Mitglied des Vereins bekannt gegebenen Kontaktdaten gerichtet ist.
% auch hier gilt: Textform = E-Mail oder andere dauerhafte Wiedergabe in Schriftzeichen
% Kontaktdaten = letzte bekannte E-Mailadresse oder Handynummer (SMS)
\item Jedes stimmberechtigte Vereinsmitglied hat das gleiche Stimmgewicht.
\item Stimmberechtigt ist jedes anwesende, aktive Mitglied.
\item Über die Beschlüsse der Mitgliederversammlung ist Protokoll zu führen, das vom Versammlungsleiter zu unterzeichnen ist.
\item Die Mitgliederversammlung als das oberste beschlussfassende Vereinsorgan ist grundsätzlich für alle Aufgaben zuständig, sofern bestimmte Aufgaben gemäß dieser Satzung nicht einem anderen Vereinsorgan übertragen wurden. Ihr sind insbesondere die Jahresrechnung und der Jahresbericht zur Beschlussfassung über die Genehmigung und die Entlastung des Vorstandes schriftlich vorzulegen.
\item Die Mitgliederversammlung entscheidet z. B. auch über
% "entscheidet z.B. auch" => optional
\begin{enumerate}
\item Höhe und Fälligkeit von Beiträgen und Umlagen,
\item Aufnahme von Darlehen ab EUR 313,37
\item Wahl und Abberufung von Vorstandsmitgliedern,
\item Wahl der Kassenprüfer,
% Ernennung von Fördermitgliedern
\item Satzungsänderungen,
\item Auflösung des Vereins.
\end{enumerate}
\item Eine Vertretung eines abwesenden Mitglieds durch ein anderes ist möglich, wenn die Vertretungsbefugnis in Textform nachgewiesen wird. Jedes anwesende Mitglied kann höchstens zwei abwesende Mitglieder vertreten.
% Wir haben uns explizit auf Textform anstelle schriftlicher Form geeinigt, um die Latte niedrig anzusetzen
% Ein abwesendes Mitglied kann ja im Nachhinein immer noch widersprechen
\item Jede satzungsmäßig einberufene Mitgliederversammlung wird als beschlussfähig anerkannt ohne Rücksicht auf die Zahl der erschienenen Vereinsmitglieder.
\item Die Mitgliederversammlung fasst ihre Beschlüsse mit einfacher Mehrheit. Bei Stimmengleichheit gilt ein Antrag als abgelehnt.
% Es hängt davon ab, wie der Antrag formuliert ist -> Wenn man einen Antrag stellt, dass etwas .....
\end{enumerate}

\section{Beurkundung von Beschlüssen}
% Wir haben uns entschieden das nicht in die Geschäftsordnung auszulagern, sondern in die Satzung zu tun um vorsätzliche Dokumentationslücken (Beispielszenario: Vorgaben vor Abstimmung kurz ändern, dann abstimmen und hinterher wieder zurück ändern) auszuschließen. Also damit eventuelle Änderungen sauber dokumentiert sind und nicht einfach mal eben (z.B. bei Geschäftsordnung des Vorstandes - die ja eh optional ist - durch Vorstand) geändert werden können.
\begin{enumerate}
\item Die in Vorstandssitzungen erfassten Beschlüsse sind schriftlich niederzulegen und von zwei Vorstandsmitgliedern zu unterzeichnen.
% Vorstand = von allen Vorstandsmitgliedern
\item Über die Beschlüsse der Mitgliederversammlung ist ein Protokoll aufzunehmen, das vom jeweiligen Versammlungsleiter und dem Protokollführer zu unterzeichnen ist.
\item Die Protokolle sollen folgende Feststellungen enthalten:
\begin{enumerate}
\item Ort und Zeit der Versammlung,
\item die Tagesordnung,
\item den Versammlungsleiter,
\item den Protokollführer,
\item die Zahl der erschienenen Mitglieder und der abgetretenen Stimmen,
\item die einzelnen Abstimmungsergebnisse der TOPs inklusive der Art der Abstimmung.
\end{enumerate}
\end{enumerate}

\section{Satzungsänderung}
\begin{enumerate}
\item Für Satzungsänderungen ist eine 3/4-Mehrheit der abgegebenen gültigen Stimmen erforderlich. Stimmenthaltungen bleiben außer Betracht.
% "Stimmenthaltungen bleiben außer Betracht" => die 3/4 beziehen sich nur auf alle Nicht-Enthaltenen Stimmen, was dazu führen kann, dass
\item Für Änderungen des Vereinszwecks gilt eine Einladungsfrist von mind. 4 Wochen. Es ist eine Mehrheit von neun zehnteln der stimmberechtigten Vereinsmitglieder erforderlich. Bei Verhinderung eines stimmberechtigten Mitgliedes kann dessen Stimmabgabe auch in Schriftform an die Vereinsanschrift bis zum Vortag um 18:00 Uhr erfolgen. In letzterem Fall gilt das Posteingangsdatum.
% Just because :P
% Bei Änderung des Vereinszwecks: Hier bezieht sich die Mehrheitsangabe explizit nicht auf alle anwesenden. Stattdessen müssen wirklich mind. 90% erscheinen und zustimmen oder sich zumindest vertreten lassen (sei es durch ein anderes Mitglied?? -> ) bzw. durch einen Brief, der bis zum Vortrag um 18:00 Uhr eingegangen sein muss.
% Eine Änderung des Vereinszwecks kommt so ziemlich einer Neugründung gleich. Bei einer Änderung des Vereinszwecks
\item Über Satzungsänderungen kann in der Mitgliederversammlung nur abgestimmt werden, wenn auf diesen Tagesordnungspunkt bereits in der Einladung zur Mitgliederversammlung hingewiesen wurde und der Einladung sowohl der bisherige als auch der vorgeschlagene neue Satzungstext beigefügt worden waren.
\item Satzungsänderungen, die von Aufsichts-, Gerichts- oder Finanzbehörden aus formalen Gründen verlangt werden, kann der Vorstand von sich aus vornehmen. Diese Satzungsänderungen müssen allen Vereinsmitgliedern zeitnah schriftlich mitgeteilt werden.
\item Der Vorstand wir ermächtigen die Satzung bei Beanstandungen durch das Registergericht oder das Finanzamt durch Vorstandsbeschluss entsprechend abzuändern.
% Vorschlag vom Vereinsregister / Finanzamt
\end{enumerate}

\section{Vereinsordnungen}
\begin{enumerate}
\item Soweit die Satzung nicht etwas Abweichendes regelt ist der geschäftsführende Vorstand ermächtigt durch Beschluss nachfolgende Ordnungen zu erlassen:
\begin{enumerate}
\item Gebührenordnung
\item Geschäftsordnung für den Vorstand.
\end{enumerate}
% (siehe hierzu auch \customref{Beschluss einer Beitragsordnung})
\end{enumerate}

% Haftung des Vereins
% https://www.haufe.de/recht/weitere-rechtsgebiete/allg-zivilrecht/wann-haften-die-vereinsmitglieder-mit-ihrem-privatvermoegen_208_76852.html

\section{Auflösung des Vereins und Vermögensbindung}
\begin{enumerate}
\item Die Auflösung des Vereins kann nur in einer zu diesem Zweck einberufenen Mitgliederversammlung beschlossen werden. Zur Auflösung des Vereins ist eine Mehrheit von drei Viertel der abgegebenen gültigen Stimmen erforderlich.
\item Sofern die Mitgliederversammlung nicht anderes beschließt, sind im Falle der Auflösung die Vorstandsmitglieder als die Liquidatoren des Vereins bestellt.
\item Bei Auflösung des Vereins oder bei Wegfall steuerbegünstigter Zwecke fällt das Vermögen des Vereins an eine oder mehrere juristische Person(en) des öffentlichen Rechts oder andere steuerbegünstigte(n) Körperschaft(en), welche es unmittelbar und ausschließlich für gemeinnützige, mildtätige oder kirchliche Zwecke zu verwenden hat. Die Begünstigte(n) wird/werden von der Mitgliederversammlung bestimmt.
\item Die Auflösung wird dem Vereinsregister durch die Liquidatoren mitgeteilt.
% ist deren Pflicht
\end{enumerate}

\section{Gültigkeit dieser Satzung}
\begin{enumerate}
\item Diese Satzung wurde durch die Mitgliederversammlung am \MyDate~beschlossen
\item Diese Satzung tritt mit Eintragung in das Vereinsregister in Kraft.
\item Alle bisherigen Satzungen treten zu diesem Zeitpunkt damit außer Kraft.
\item Wird ein Paragraph der Satzung für ungültig erklärt, bleibt die Gültigkeit der anderen Satzungsparagraphen davon unberührt.
\end{enumerate}
\end{document}
